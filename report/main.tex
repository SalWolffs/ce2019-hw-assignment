\documentclass[a4paper,twoside]{article}

\input{preamble.tex}

\newcommand{\atom}[1]{\mbox{\texttt{#1}}}

\usepackage{rwd-drafting}

\title{Cryptographic Engineering ECC Project}

\author{Sal Wolffs, Lars Jellema}

\begin{document}
\maketitle

\section*{Exercise 4} 
\atom{modaddn.vhd} was written by copying \atom{modadd4.vhd} and replacing any
occurrences of \atom{3} and \atom{4} by \atom{n-1} resp. \atom{n}, as per the
example of \atom{addn.vhd} and \atom{add4.vhd}. Expanding the test bench was
done by adjusting the default width of the generic \atom{tb\_modaddn} entity,
taking some 128-bit random numbers for \atom{a\_i}, \atom{b\_i} and \atom{p\_i},
making sure one set requires a subtraction and one doesn't, and computing
appropriate values for \atom{sum\_true} in python.

\section*{Exercise 7}
For \atom{modaddn\_mult.vhd}, we wrote a separate file for the finite state
machine (TODO: add full paragraph on \atom{ctr\_fsm}) and added it and
\atom{modaddn} as components. \atom{modaddsubn} was not needed here, we know
we'll only do additions. We adjusted the FSM to hold "done" after reaching its
limit, rather than resetting. 

\section*{Exercise 8}
We wrote a modular piso left shift register (i.e. one that outputs the MSB and
shifts it out on each cycle) \atom{piso\_lshiftreg} to scan through the bits of
\atom{b} in \atom{modmultn}. We also wrote a modulo left shifter which shifts
its input by 1 bit and reduces it modulo the given \atom{p}, using the same
logic as in \atom{modaddn}. Re-using \atom{ctr\_fsm}, this time with the width
of \atom{b} as constant input (TODO: might be able to optimise by hardcoding
within \atom{ctr\_fsm}, at the cost of losing this re-use), we could now easily
double-and-add modulo p within the main module.

\section*{Exercise 9}
TODO: Drawing. The basic design is a simple glueing of earlier components with
some muxes and input buffers: \atom{modmultn} has buffered input and only
updates its inputs on receiving "start".  \atom{modaddsubn} is asynchronous, and
so will compute whenever inputs change.  In the interest of keeping energy
consumption low, we buffer the inputs of \atom{modaddsubn} as well in the
\atom{modarithn} unit. All commands except "01" are done as soon as \atom{start}
is released (actually, they're done on the next rising edge after clock, but
coding that ran into vhdl syntax issues). So a mux selects \atom{done} from
either the \atom{done} output of the multiplier (on command "01") or otherwise
just outputs \atom{not start}. Output is chosen by mux based on the command from
either \atom{modmultn}, \atom{modaddsubn}, or just the \atom{a} input (on
\atom{NOP}) based on command received. The \atom{cmd\_reg} allows us to adjust
the design to deal with the \atom{command} input becoming unstable after
\atom{start} goes low.




\end{document}
