\documentclass[a4paper,twoside]{article}

\input{preamble.tex}

\newcommand{\atom}[1]{\mbox{\texttt{#1}}}

\usepackage{rwd-drafting}

\title{Cryptographic Engineering ECC Project}

\author{Sal Wolffs, Lars Jellema}

\begin{document}
\maketitle

\section*{Exercise 4} 
\atom{modaddn.vhd} was written by copying \atom{modadd4.vhd} and replacing any
occurrences of \atom{3} and \atom{4} by \atom{n-1} resp. \atom{n}, as per the
example of \atom{addn.vhd} and \atom{add4.vhd}. Expanding the test bench was
done by adjusting the default width of the generic \atom{tb\_modaddn} entity,
taking some 128-bit random numbers for \atom{a\_i}, \atom{b\_i} and \atom{p\_i},
making sure one set requires a subtraction and one doesn't, and computing
appropriate values for \atom{sum\_true} in python.





\end{document}
